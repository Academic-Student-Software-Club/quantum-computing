\documentclass{article}

\usepackage{amsmath}
\usepackage{graphicx}
\usepackage{wrapfig}
\usepackage[colorlinks=true, allcolors=blue]{hyperref}
\usepackage[letterpaper,top=2cm,bottom=2cm,left=1.75cm,right=1.75cm,marginparwidth=1.75cm]{geometry}
\usepackage[english]{babel}
\usepackage{float}
\usepackage{blindtext}
\usepackage{mathtools}
\DeclarePairedDelimiter\bra{\langle}{\rvert}
\DeclarePairedDelimiter\ket{\lvert}{\rangle}
\usepackage[colorlinks=true, allcolors=blue]{hyperref}
\usepackage{xcolor}
\renewcommand{\thesubsection}{\thesection.\alph{subsection}}
\usepackage{gensymb}
\usepackage{tabularx}
% chktex 44


\newcolumntype{b}{X}
\newcolumntype{s}{>{\hsize=.25\hsize}X}


\title{Q. Comp HW 6}
\author{Michael Bibbs}

\begin{document}
	\maketitle
	\section{Problem 1}
Consider a Fourier Transform
\[\ket{j}\rightarrow\frac{1}{\sqrt{N}}\sum_{k=0}^{N-1}e^{2\pi i j k/N}\ket{k}\]
\subsection{}
It is (practically) given that the Quantum Fourier Transform is unitary, since it can be decomposed into a product of other unitary matrices. Therefore, if we let the unitary representation of the QFT be $U$, then there exists another unitary $U^\dagger$ such that $UU^\dagger=I$.
\\ So, 
\[U=\frac{1}{\sqrt{N}}\sum_{k=0}^{N-1}e^{2\pi i j k/N}\longrightarrow U^*=\frac{1}{\sqrt{N}}\sum_{k=0}^{N-1}e^{-2\pi i j k/N}\longrightarrow \boxed{U^{*T}=U^\dagger=\frac{1}{\sqrt{N}}\sum_{k=0}^{N-1}e^{-2\pi i j k/N}}\]
In the form of the sum, switching the indexes $j$ and $k$ do nothing, as they are multiplied together. This matrix is symmetric precisely due to this product (more importantly, the matrix is square.), so the transpose shouldn't do anything \textit{here} anyways- only taking the conjugate. 
\\ Since we started from the fact that the QFT is unitary, it should be no surprise that consecutive application of both the QFT and Inverse QFT give the identity operator.
\subsection{}
For three qubits, the circuits corresponding to the QFT and QFT$^{-1}$ are:
\begin{figure}[h]
	\centering
	\includegraphics[width=0.7\textwidth]{fig2.png}
	\caption{Quantum circuit with $U=$ QFT, and $U^\dagger=\text{ QFT}^{-1}$}
\end{figure}
\pagebreak
\\ We can find the 8 $\times$ 8 unitary operator $U$ by directly evaluating it's closed form expression:
\[U\ket{j}=\frac{1}{\sqrt{N}}\sum_{k=0}^{N-1}e^{2\pi i j k/N}\ket{k}\]
\[\begin{array}{c|rrrrrrrr}
	\ket{j} & \ket{k_0} &\ket{k_1} &\ket{k_2} &\ket{k_3} &\ket{k_4} &\ket{k_5} &\ket{k_6} &\ket{k_7} \\
	\hline\hline
	\ket{0} & 1\ket{0}&1\ket{1}&1\ket{2}&1\ket{3}&1\ket{4}&1\ket{5}&1\ket{6}&1\ket{7}\\

	\ket{1} & 1\ket{0}&e^{\frac{2 \pi  i}{8}}&e^{\frac{2 \pi  i 2}{8}}&e^{\frac{2 \pi  i 3}{8}}&e^{\frac{2 \pi  i 4}{8}}&e^{\frac{2 \pi  i 5}{8}}&e^{\frac{2 \pi  i 6}{8}}&e^{\frac{2 \pi  i 7}{8}}\\
	\ket{2} & 1\ket{0}&e^{\frac{2 \pi  i 2}{8}}&e^{\frac{2}{8} \pi  i 2\ 2}&e^{\frac{2}{8} \pi  i 2\ 3}&e^{\frac{2}{8} \pi  i 2\ 4}&e^{\frac{2}{8} \pi  i 2\ 5}&e^{\frac{2}{8} \pi  i 2\ 6}&e^{\frac{2}{8} \pi  i 2\ 7}\\
	\ket{3} & 1\ket{0}&e^{\frac{2 \pi  i 3}{8}}&e^{\frac{2}{8} \pi  i 3\ 2}&e^{\frac{2}{8} \pi  i 3\ 3}&e^{\frac{2}{8} \pi  i 3\ 4}&e^{\frac{2}{8} \pi  i 3\ 5}&e^{\frac{2}{8} \pi  i 3\ 6}&e^{\frac{2}{8} \pi  i 3\ 7}\\
	\ket{4} &1\ket{0}&e^{\frac{2 \pi  i 4}{8}}&e^{\frac{2}{8} \pi  i 4\ 2}&e^{\frac{2}{8} \pi  i 4\ 3}&e^{\frac{2}{8} \pi  i 4\ 4}&e^{\frac{2}{8} \pi  i 4\ 5}&e^{\frac{2}{8} \pi  i 4\ 6}&e^{\frac{2}{8} \pi  i 4\ 7}\\
	\ket{5} &1\ket{0}&e^{\frac{2 \pi  i 5}{8}}&e^{\frac{2}{8} \pi  i 5\ 2}&e^{\frac{2}{8} \pi  i 5\ 3}&e^{\frac{2}{8} \pi  i 5\ 4}&e^{\frac{2}{8} \pi  i 5\ 5}&e^{\frac{2}{8} \pi  i 5\ 6}&e^{\frac{2}{8} \pi  i 5\ 7}\\
	\ket{6} & 1\ket{0}&e^{\frac{2 \pi  i 6}{8}}&e^{\frac{2}{8} \pi  i 6\ 2}&e^{\frac{2}{8} \pi  i 6\ 3}&e^{\frac{2}{8} \pi  i 6\ 4}&e^{\frac{2}{8} \pi  i 6\ 5}&e^{\frac{2}{8} \pi  i 6\ 6}&e^{\frac{2}{8} \pi  i 6\ 7}\\
	\ket{7} & 1\ket{0}&e^{\frac{2 \pi  i 7}{8}}&e^{\frac{2}{8} \pi  i 7\ 2}&e^{\frac{2}{8} \pi  i 7\ 3}&e^{\frac{2}{8} \pi  i 7\ 4}&e^{\frac{2}{8} \pi  i 7\ 5}&e^{\frac{2}{8} \pi  i 7\ 6}&e^{\frac{2}{8} \pi  i 7\ 7}\\
\end{array}
\]
Writing the QFT as a full 8 $\times$ 8 unitary,
\[\boxed{U=\dfrac{1}{\sqrt{8}}
\left(
\begin{array}{cccccccc}
	1 & 1 & 1 & 1 & 1 & 1 & 1 & 1 \\
	1 & e^{\frac{i \pi }{4}} & i & e^{\frac{3 i \pi }{4}} & -1 & e^{-\frac{1}{4} (3 i \pi )} & -i & e^{-\frac{1}{4} (i \pi )} \\
	1 & i & -1 & -i & 1 & i & -1 & -i \\
	1 & e^{\frac{3 i \pi }{4}} & -i & e^{\frac{i \pi }{4}} & -1 & e^{-\frac{1}{4} (i \pi )} & i & e^{-\frac{1}{4} (3 i \pi )} \\
	1 & -1 & 1 & -1 & 1 & -1 & 1 & -1 \\
	1 & e^{-\frac{1}{4} (3 i \pi )} & i & e^{-\frac{1}{4} (i \pi )} & -1 & e^{\frac{i \pi }{4}} & -i & e^{\frac{3 i \pi }{4}} \\
	1 & -i & -1 & i & 1 & -i & -1 & i \\
	1 & e^{-\frac{1}{4} (i \pi )} & -i & e^{-\frac{1}{4} (3 i \pi )} & -1 & e^{\frac{3 i \pi }{4}} & i & e^{\frac{i \pi }{4}} \\
\end{array}
\right) }\]
In this form, we can easily find the QFT$^{-1}$ by taking the conjugate-transpose:
\[\boxed{U^\dagger=\dfrac{1}{\sqrt{8}}
\left(
\begin{array}{cccccccc}
	1 & 1 & 1 & 1 & 1 & 1 & 1 & 1 \\
	1 & e^{-\frac{1}{4} (i \pi )} & -i & e^{-\frac{1}{4} (3 i \pi )} & -1 & e^{\frac{3 i \pi }{4}} & i & e^{\frac{i \pi }{4}} \\
	1 & -i & -1 & i & 1 & -i & -1 & i \\
	1 & e^{-\frac{1}{4} (3 i \pi )} & i & e^{-\frac{1}{4} (i \pi )} & -1 & e^{\frac{i \pi }{4}} & -i & e^{\frac{3 i \pi }{4}} \\
	1 & -1 & 1 & -1 & 1 & -1 & 1 & -1 \\
	1 & e^{\frac{3 i \pi }{4}} & -i & e^{\frac{i \pi }{4}} & -1 & e^{-\frac{1}{4} (i \pi )} & i & e^{-\frac{1}{4} (3 i \pi )} \\
	1 & i & -1 & -i & 1 & i & -1 & -i \\
	1 & e^{\frac{i \pi }{4}} & i & e^{\frac{3 i \pi }{4}} & -1 & e^{-\frac{1}{4} (3 i \pi )} & -i & e^{-\frac{1}{4} (i \pi )} \\
\end{array}
\right)}\]
Although we start with the knowledge the QFT is unitary, we can check this by finding $UU^\dagger$.
\[UU^\dagger=
\left(
\begin{array}{cccccccc}
	1 & 0 & 0 & 0 & 0 & 0 & 0 & 0 \\
	0 & 1 & 0 & 0 & 0 & 0 & 0 & 0 \\
	0 & 0 & 1 & 0 & 0 & 0 & 0 & 0 \\
	0 & 0 & 0 & 1 & 0 & 0 & 0 & 0 \\
	0 & 0 & 0 & 0 & 1 & 0 & 0 & 0 \\
	0 & 0 & 0 & 0 & 0 & 1 & 0 & 0 \\
	0 & 0 & 0 & 0 & 0 & 0 & 1 & 0 \\
	0 & 0 & 0 & 0 & 0 & 0 & 0 & 1 \\
\end{array}
\right)
=I\]


\end{document}
